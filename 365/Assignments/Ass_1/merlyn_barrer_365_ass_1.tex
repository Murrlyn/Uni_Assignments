%This is my super simple Real Analysis Homework template

\documentclass[12pt]{article}
\usepackage[utf8]{inputenc}
\usepackage[english]{babel}
\usepackage[margin=2.5cm]{geometry}
\usepackage[]{amsmath}
\usepackage[]{amsmath, mathtools}
\usepackage[]{amssymb} %gives us the character \varnothing
\usepackage{xcolor}
\usepackage{xparse}
\usepackage[italicdiff]{physics}
\usepackage{parskip}


% definition of the "problem" environment.
\newsavebox{\problembox}
\newcounter{problem}[section]

\newenvironment{problem}[1][]
    {\refstepcounter{problem}\par\medskip\noindent
    \textbf{Problem~\theproblem. #1} \rmfamily \em} %chktex 6
    {\medskip}

% Definition of the "subproblem" environment.
\newsavebox{\subproblembox}
\newcounter{subproblem}[problem]

\newenvironment{subproblem}[1][]
    {\refstepcounter{subproblem}\par\medskip\noindent
    \textbf{(\roman{subproblem}) #1} \rmfamily \em} %chktex 6
    {\par}

% Notation for cyclic groups
\newcommand{\cyc}[1]{\langle#1\rangle}

\newcommand{\set}[1]{\left\{ #1 \right\}}
\NewDocumentCommand\seq{O{1}O{\infty}mm}
    {\set{#3_#4}_{#4=#1}^{#2}}

\DeclareMathOperator{\lcm}{lcm}

% Blackboard notation
\newcommand{\R}{\mathbb{R}}
\newcommand{\N}{\mathbb{N}}
\newcommand{\Z}{\mathbb{Z}}
\newcommand{\Q}{\mathbb{Q}}
\newcommand{\C}{\mathbb{C}}

\newcommand{\X}{\mathbb{X}}

\title{MATH343 Assignment 9}
\author{Merlyn Barrer 33293451}
%This information doesn't actually show up on your document unless you use the maketitle command below

\begin{document}
\maketitle %This command prints the title based on information entered above

Let \(E, F\) be subsets of a metric space \((X, \rho)\).
\begin{subproblem}
    If \(E \subseteq F\), show that \(\overline{E} \subseteq \overline{F}\).
\end{subproblem}

Given arbitrary \(x \in \overline E\), by Theorem 3.16, there exists a convergent sequence \(\set{x_n} \subseteq{E}\) such that \(x_n \to x\) .
\(E \subseteq F\), so we have \(\set{x_n} \subseteq F\).
Making use of Theorem 3.16 again, this implies \(x \in \overline{F}\).
Therefore \(E \subseteq F\).

\begin{subproblem}
    Prove that \(\overline{E \cap F} \subseteq \overline{E} \cap \overline{F}\).
    In \((\R, d)\) give an example of two sets \(E_1\) and \(E_2\) such that
    \begin{equation*}
        \overline{E_1 \cap E_2} \neq \overline{E_1} \cap \overline{E_2}.
    \end{equation*}
\end{subproblem}


Note that \(E \subseteq \overline E\) and \(F \subseteq \overline F\).
Therefore \(E \cap F \subseteq \overline{E} \cap \overline F\).
By (i), this implies that \(\overline{E \cap F} \subseteq \overline{\overline{E} \cap \overline{F}}\).
\(\overline{E}\) and \(\overline{F}\) are closed, so \(\overline{E} \cap \overline{F}\) is closed also.
Therefore \(\overline{\overline{E} \cap \overline{F}} = \overline{E} \cap \overline{F}\).
Therefore \(\overline{E \cap F} \subseteq \overline{E} \cap \overline{F}\).

Take \(E_1 = (69, 6969)\) and \(E_2 = (6969, 696969)\).
Then
\begin{gather*}
    \overline{E_1 \cap E_2} = \overline{\emptyset} = \emptyset,\\
    \overline{E_1} \cap \overline{E_2} = [69, 6969] \cap [6969, 696969] = \set{6969}.
\end{gather*}
So \begin{equation*}
    \overline{E_1 \cap E_2} \neq \overline{E_1} \cap \overline{E_2}.
\end{equation*}


\end{document} %chktex 17,